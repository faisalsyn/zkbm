\section{Experimental evaluation}\label{sec:eval}
In this section we establish a baseline performance for zookeeper on our
testbed and then provide latency and throughput numbers for the selected
synchronization primitives.

Baseline numbers for zookeeper are obtained using smoketest [ref]. We explore
the performance of test-and-set locks, queues and barriers implemented in Java.

We first set up zookeeper with 1, 3 and 5 processes on a single machine and
observe the experiments. Then, we fan out the processes to distict machines and
measure the impact of added network latency on basic operations and the
synchronization primitives. (Finally, we repeat the experiments on
geographically distributed nodes in EC2)

\subsection{Test bed}
\note{multiple physical machines}
The testbed consists of 5 quad-core Xeon (type and clock) machines with commodity 500GB 7.2k rpm disks. They are interconnected with 1Gbps Ethernet on a single switch.

Average RTT for pings is (ping lat)
Average network throughput is (throughput=

\subsection{Baseline performance}
\note{smoketest and baseline} In this section we will perform two experiments to establish a baseline for later sections. We would like to establish limits on the system. These limits represent workloads and environment conditions that will saturate the system. Workload is represented by the number of clients, number of requests per second, and the type of requests. Environment conditions are the number of zookeeper servers and condition of links connecting them.



\subsection{Request latency}
\note{baseline experiment. effect of adding machines, effect of adding zookeeper servers.}

\subsection{synchronization primitives}
\note{test test-and-set and queues (as in paper: reactive synchronization}

\subsection{application performance}
\note{map reduce. effect of adding machines, effect of adding zookeeper servers.}

