Synchronization in distributed systems is a well-known issue, yet the recent towards geo distributed systems with strongly consistency requirements raises new issues. Systems operated on global scale deal with large distances and heterogeneous network topologies which lead to network roundtrip times in the order of 100ms and more. Existing best-practice "receipes" for distributed synchronization implemented on top of plattforms such a Zookeeper are particularly sensitive to round-trip latency. They are designed for managing contention, but cause long wait-times even when there is not any competition for resources. In this paper we analyze the impact of communication latency on synchronization primitives implemented on top of Zookeeper in a real-world environment. By varying the round-trip-times between clients and znodes, testing different levels of contention and modelling the messaging patterns we characterize synchonization primitives' behavior. We show that the optimal choice of existing primitives depends on the specific use-case we identify potential for improvement. Based on our findings we propose optimized asynchonous designs for queue and lock primitives which gracefully handle high latency. Our evaluation shows that the proposed designs  provide an advantage under low latency conditions and outperform under high latency.