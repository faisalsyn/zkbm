Synchronization in distributed systems is well-studied. Yet the recent trend towards geo-distributed systems with strong consistency requirements raises new challenges. Systems operating on global scale deal with long network round-trip times of 100ms and more. Existing best-practice solutions for distributed synchronization implemented on top of plattforms such a Zookeeper are particularly sensitive to communication latency. They are designed to manage contention, but cause long delays regardless of contention on resources. In this paper we analyze the impact of communication latency on synchronization primitives implemented on top of Zookeeper in a real-world environment. By varying round-trip times and contention levels we characterize the synchonization primitives' behavior. Our experimental study reveals that queues are better suited for high contention environments, whereas locks perform better in a low contention, high latency setting. Based on our observation we propose an optimized design for queues and locks which gracefully handles high latency.
