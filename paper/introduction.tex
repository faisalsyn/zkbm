\section{Introduction}\label{sec:intro}

Coordination is essential for large-scale distributed applications. Handling the simplest operations in a distributed manner gave rise to unprecedented challenges. Different forms of coordination are used to handle a variety of tasks. Distributed Transactions are one example, leader election and group membership another. Worrying about aspects of synchronization, concurrency, and distributed management is a heavy burden on application developers. This is why distributed consensus protocols were designed and implemented to be leveraged by those developers, such as ZooKeeper\cite{zookeeper} and Chubby\cite{chubby}.

Distributed synchonization and consensus come at a cost however and have been an emphasis of research for a long time. Numerous papers look into reducing latency locally on a single machine through dynamic choice of primitives\cite{reactive}, heuristics such as lock elision\cite{lock_elison} or hardware support for transactional memory\cite{transactional_memory}. A different branch of research looks into distributed synchonization through protocols like two phase commit\cite{two_phase_commit} or Paxos\cite{paxos} in a search for fault-tolerance and graceful behavior under high contention. While the first approach provides low latency solutions, it does not scale across a local network or geo distributed datacenters. The latter approach provides fault-tolerance and predictable performance but comes with substantial latency overhead in the first place. The current emphasis on strongly consistent geo distributed application aggravates this latency issue. For example, Google recently revealed the distributed F-1 database system \cite{google_f1} which provides strong consistency shows higher latencies than its predecessor and pushes complexity down to the client API to mask some of this additional cost.

Here is a need to reinvestigate synchronization protocols for large-scale distributed systems. In these systems communication latency depends on network round trip times and can reach hundreds of milliseconds. This dramatic difference to the conventional multiprocessor environment might carry with it new revelations on the community's prejudice on traditional synchronization protocols. Existing distributed consensus packages deliver coordination schemes to developers and typically provide proven "recipes" \cite{zookeeper_recipes} for building synchronization mechanisms. For example, ZooKeeper provides an API to manipulate hierarchically organized wait-free data objects, resembling a file system. Using its atomic broadcast protocol Zab\cite{ZAB} these manipulations are guaranteed to be FIFO ordered which enables developers to quickly build more complex synchronization mechanisms on top. Although ease of of use is an advantage upfront, the inefficient implementation or inappropriate use of these low level primitives later translates into lower system throughput.

In this paper we take first steps towards improving synchronization in geo-distributed settings and show that the practical performance of synchronization mechanisms heavily depends on the use-case. For our experiments we focus on providing mutual exclusion for a resource with varying latency for client-server and server-servers communication at different levels of contention. We investigate two primitives, queues and locks with two different implementations each. The implementations, synchonous and asynchronous, are inspired by traditional distributed systems "recipies" and approaches taken in parallel hardware architectures. While our results show familiar results for advantages of locks under low contention and queues under high load, the varying range of round trip times shows non-obvious benefits of asynchronous implementations.

The rest of the paper is organized as follows. Section~\ref{sec:framework} describe the framework of our experiments and overview basic concepts and technologies used. A mathematical analysis is presented in Section~\ref{sec:analysis} where we provide a model of observed latency. Experimental results are then displayed in Section~\ref{sec:eval}. Finally, the paper concludes with a summary and future directions in Section~\ref{sec:conclusions}.